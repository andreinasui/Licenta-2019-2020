%!TEX spellcheck=ro_RO
%!TEX root = ./main.tex
\chapter{Concluzii}\label{ch:4concluzii}
Lucrarea demonstrează conceptul transpunerii informațiilor conținute în semnalele EEG sub forma unor imagini alb-negru și clasificarea acestora în trei clase/stări mentale diferite folosind rețelele neuronale convoluționale. Pentru realizarea proiectului au fost parcurse etape pentru extragerea și prelucrarea datelor, construirea unui set de date pe baza acestora, realizarea unei rețele convoluționale și antrenarea acesteia folosind setul de date creat anterior. Rezultatele promițătoare obținute aplicând aceste principii dovedesc utilitatea folosirii rețelelor convoluționale pentru clasificare, dar și ideea transpunerii datelor sub forma unor imagini.

Extragerea semnalelor EEG a fost realizată folosind casca Muse 2016, având disponibili patru electrozi EEG. Datele extrase folosind această cască au un raport zgomot-semnal util mai mare decât variantele medicale specializate. Această inconveniență poate duce la pierderea informațiilor specifice anumitor stări mentale, acestea fiind apoi clasificate greșit. Avantajele oferite de această cască însă sunt  prețul relativ mic, accesibilitatea și ușurința de lucru. Folosirea altor soluții hardware comparative, cu un număr crescut de electrozi și cu raport zgomot-semnal util mai mic va duce la o înregistrare mai corectă a stărilor mentale rezultând o clasificare finală mai bună.

În sensul îmbunătățirii atât a clasificatorului, cât și a metodelor de procesare și creare a setului de date, se propun următoarele idei pentru continuarea studiului:
\begin{itemize}
\item În \autoref{tabel:classification-report} au fost prezentate valorile metricilor modelului creat, iar mai apoi, pe baza acestora, împreună cu datele prezentate în \autoref{fig:conf_matrix}, a fost prezentată confuzia modelului în clasificarea stărilor \textit{neutru} și \textit{relaxat}, creată de modalitățile foarte asemănătoare de înregistrare a acestori stări. Pentru remedierea acestor deficiente ale rețelei, este necesară diferențierea claselor \textit{neutru} și \textit{relaxat} în momentul înregistrării semnalelor EEG prin folosirea unor metode distinctive
\item Gruparea atributelor în funcție de corelația acestora și așezarea într-o anumită ordine. Este posibil ca prin acest pas să fie evidențiate mai ușor tipare reprezentative stărilor mentale
\item Extragerea unui număr mai mare de atribute din semnalele EEG, rezultând într-o dimensiune mai mare a matricei/imaginii create. Având o imagine mai mare, tiparele reprezentative stărilor mentale vor fi mai evidente, crescând acuratețea rețelei
\end{itemize}