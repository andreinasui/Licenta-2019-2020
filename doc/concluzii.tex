%!TEX spellcheck=ro_RO
%!TEX root = ./main.tex
\chapter{Concluzii}\label{ch:4concluzii}
Lucrarea demonstrează conceptul transpunerii informațiilor conținute în semnalele EEG sub forma unor imagini alb-negru și clasificarea acestora în trei clase/stări mentale diferite folosind rețelele neuronale convoluționale. Rezultatele promițătoare obținute aplicând aceste principii dovedesc utilitatea folosirii rețelelor convoluționale pentru clasificare, dar și ideea transpunerii datelor sub forma unor imagini.

Pentru extragerea semnalelor EEG a fost folosită casca comercial valabilă Muse 2016 având disponibili patru electrozi EEG. Datele extrase folosind această cască au un raport zgomot-semnal util mai mare decât variantele medicale specializate. Această inconveniență poate duce la pierderea informațiilor specifice anumitor stări mentale, acestea fiind apoi clasificate greșit. Folosind alte soluții hardware comparative, cu un număr crescut de electrozi și cu raport zgomot-semnal util mai mic, poate duce la o înregistrare mai corectă a stărilor mentale rezultând o clasificare finală mai corectă.

În sensul îmbunătățirii atât a clasificatorului, cât și a metodelor de procesare și creare a setului de date, se propun următoarele idei pentru continuarea studiului:
\begin{itemize}
\item Separarea optimă a claselor neutru și relaxat în momentul înregistrării semnalelor EEG aferente. În \autoref{tabel:classification-report} a fost prezentată confuzia modelului creată de modalitățile foarte asemănătoare de înregistrare a acestori stări
\item Extragerea unui număr mai mare de atribute din semnalele EEG, rezultând într-o dimensiune mai mare a matricei/imaginii create
\item Gruparea atributelor în funcție de corelația acestora și așezarea într-o anumită ordine. Este posibil ca prin acest pas să fie evidențiate mai ușor tipare reprezentative stărilor mentale
\end{itemize}